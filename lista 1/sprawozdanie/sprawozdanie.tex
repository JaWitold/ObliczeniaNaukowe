\documentclass[11pt, a4paper]{article}
\usepackage{polski}
\usepackage[utf8]{inputenc}
\usepackage{xcolor}
\usepackage{listings}

\definecolor{mGreen}{rgb}{0,0.6,0}
\definecolor{mGray}{rgb}{0.5,0.5,0.5}
\definecolor{mPurple}{rgb}{0.58,0,0.82}
\definecolor{backgroundColour}{rgb}{0.95,0.95,0.95}

\lstdefinestyle{CStyle}{
    backgroundcolor=\color{backgroundColour},   
    commentstyle=\color{mGreen},
    keywordstyle=\color{magenta},
    numberstyle=\tiny\color{mGray},
    stringstyle=\color{mPurple},
    basicstyle=\footnotesize,
    breakatwhitespace=false,         
    breaklines=false,                 
    captionpos=b,                    
    keepspaces=false,                 
    numbers=none,                    
    numbersep=5pt,                  
    showspaces=false,                
    showstringspaces=false,
    showtabs=false,                  
    tabsize=2,
    language=C
}

\title{Sprawozdanie \\
\large Obliczenia naukowe - lista 1\\}
\author{Witold Karaś}
\date{}
\begin{document}
\maketitle
\section*{zadanie 1}
Program oblicza epsilon maszynowy (\emph{macheps}), liczbę maszynową \emph{eta} oraz największą liczbę jaką można zapisać jako float. Wykonuje te operacje odpowiednio dla typów Float16, Float32 oraz Float64.\\\\
\textbf{Rozwiązanie}\\
a) epsilon maszynowy jest obliczany iteracyjnie. W pętli sprawdzana jest prawdziwość warunku $1.0 + \epsilon > 1.0$ i co iteracje $\epsilon$ jest zmiejszany o połowę.\\
Precyzja arytmetyki informuje nas o odległościach między kolejnymi liczbami w arytmetyce a epsilon maszynowy określa odległość między 1 i następną liczbą x taką że x > 1
\begin{table}[ht]
    \centering
    \begin{tabular}{|c|c|c|c|}
      \hline
      \emph{typ zmiennej} & \emph{macheps} & nextfloat & \texttt{float.h} \\
      \hline
      \hline
      Float16 & 0.000977 & 0.000977 & nie występuje \\
      Float32 & 1.1920929e-7 & 1.1920929e-7 & 1.192092896e-07F \\
      Float64 & 2.220446049250313e-16 & 2.220446049250313e-16 & 2.2204460492503131e-016\\
      \hline
    \end{tabular}
    \caption{wartości z wyjścia programu \texttt{zadanie1.jl}}
  \end{table}

b) eta obliczana jest iteracyjnie. W pętli dzielona jest przez 2 jeśli prawdziwy jest warunek $\eta/2 > 0$.\\
Eta określa doległość między zerem a następną liczbą x taką że $x > 0$ czyli jest MIN\textsubscript{SUB}. Wbudowana funkcja \textbf{floatmin} według specyfikacji zwraca najmniejszą liczbę dodatnią w postaci zmiennopozycyjnej, jednak jest ona większa od uzyskanych wyników. Wynika to z faktu że zwracana wartość przez \textbf{floatmin} jest w postaci normalnej.
  \begin{table}[ht]
    \centering
    \begin{tabular}{|c|c|c|c|}
      \hline
      \emph{typ zmiennej} & \emph{obliczona eta} & floatmin & eta \\
      \hline
      \hline
      Float16 & 6.0e-8 & 6.104e-5 & 6.0e-8 \\
      Float32 & 1.0e-45 & 1.1754944e-38 & 1.0e-45 \\
      Float64 & 5.0e-324 & 2.2250738585072014e-308 & 5.0e-324 \\
      \hline
    \end{tabular}
    \caption{wartości z wyjścia programu \texttt{zadanie1.jl}}
  \end{table}

c) wartość maksymalną dla typu zmiennopozycyjnego jest obliczana iteracyjnie poprzez mnożenie przez 2. rozpoczynamy od wartości prevfloat(one(type))ponieważ jest złożona z samych jedynek w części ułamkowej. Wykładnik podczas mnozenia przez 2 zmienia się jedynie wykładnik dzięki czemu stosunkowo szybko program znajduje wartość maksymalną.\\\\\\

 \begin{table}[ht]
    \centering
    \begin{tabular}{|c|c|c|c|}
      \hline
      \emph{typ zmiennej} & obliczona wartość & floatmax & \texttt{float.h} \\
      \hline
      \hline
      Float16 & 6.55e4 & 6.55e4 & nie występuje \\
      Float32 & 3.4028235e38 & 3.4028235e38 & 3.402823466e+38F \\
      Float64 & 1.7976931348623157e308 & 1.7976931348623157e308 & 1.7976931348623158e+308 \\
      \hline
    \end{tabular}
    \caption{wartości z wyjścia programu \texttt{zadanie1.jl}}
  \end{table}
  Poniższy fragment przedstawia definicje znalezione w pliku \verb|float.h|

  \begin{lstlisting}[style=CStyle, breaklines]
    //dla typu float
    #define FLT_EPSILON      1.192092896e-07F        // smallest such that 1.0+FLT_EPSILON != 1.0
    #define FLT_MAX          3.402823466e+38F        // max value
    #define FLT_MIN          1.175494351e-38F        // min normalized positive value
    #define FLT_TRUE_MIN     1.401298464e-45F        // min positive value
    //dla typu double
    #define DBL_EPSILON      2.2204460492503131e-016 // smallest such that 1.0+DBL_EPSILON != 1.0
    #define DBL_MAX          1.7976931348623158e+308 // max value
    #define DBL_MIN          2.2250738585072014e-308 // min positive value
    #define DBL_TRUE_MIN     4.9406564584124654e-324 // min positive value
  \end{lstlisting}

  \textbf{Wnioski}
  Znalezione przez program liczby są zgodne z wynikami funkcji wbudowanych. 
  \section*{zadanie 2}
  Program oblicza \emph{macheps} metodą Kahana. Otrzymane wyniki są, z dokładnością do znaku, zgodne z tymi które otrzymałem w zadaniu 1
  \begin{table}[ht]
    \centering
    \begin{tabular}{|c|c|}
      \hline
      \emph{typ zmiennej} & \emph{output} \\
      \hline
      Float16 &  -0.000977 \\
      Float32 &  1.1920929e-7 \\
      Float64 &  -2.220446049250313e-16 \\
      \hline
    \end{tabular}
    \caption{wartości z wyjścia programu \texttt{zadanie2.jl}}
  \end{table}

  \section*{zadanie 3}
  Program sprawdza czy eksponenty skrajnych liczb z przedziału są identyczne, jeżeli są oznacza to że liczby w przedziale są równomiernie rozmieszczone. 
  Następnie ze wzoru $ \delta = 2^{eksponent - 1023} * 2^{-52} $ jest obliczana $\delta$ która reprezentuje odstęp między liczbami w zakresie.\\
  \textbf{Wnioski:}\\
  Liczby w artytmetyce zmiennopozycyjnej nie są rozmieszczone równomiernie na osi liczbowej. Im są bliżej 0 tym są bliżej siebie, jednak nie osiągają 0 (ograniczeniem dolnym jest liczba eta z zadania 1)
  
  \begin{table}[ht]
    \centering
    \begin{tabular}{|c|c|}
      \hline
      \emph{zakres} & \emph{output} \\
      \hline
      (0.5, 1.0) & 1.1102230246251565e-16 \\
      (1.0, 2.0) & 2.220446049250313e-16 \\
      (2.0, 4.0) & 4.440892098500626e-16 \\
      \hline
    \end{tabular}
    \caption{wartości z wyjścia programu \texttt{zadanie3.jl}}
  \end{table}

  \section*{zadanie 4}

  Sprawdziłem metodą brute force - iteracyjnie po $x \in (1, 2)$ jaka jest najmniejsza liczba spełniająca nierówność $x \cdot \frac{1}{x} \neq 1$
  \\
  \emph{output: } $1.000000057228997$
  \section*{zadanie 5}
  Przetestowałem obliczanie wektora skalarnego 4 wskazanymi metodami. Przez fakt, że iloczyny były liczbami o bardzo różnym rzędzie wartości (bardzo małe i bardzo duże liczby) a precyzja arytmetyki jest ograniczona.

  \begin{table}[ht]
    \centering
    \begin{tabular}{|c|c|}
      \hline
      \emph{typ zmiennej} & \emph{output} \\
      \hline
      \hline
      Float32 & -0.4999443 \\
       & -0.4543457 \\
       & -0.5 \\
       & -0.5 \\
      \hline
      Float64 & 1.0251881368296672e-10 \\
       & -1.5643308870494366e-10 \\
       & 0.0 \\
       & 0.0 \\
      \hline
    \end{tabular}
    \caption{wartości z wyjścia programu \texttt{zadanie5.jl}}
  \end{table}
  
  \section*{zadanie 6}
  Zaprogramowałem dwie funkcje $$f(x) = \sqrt{x^2 + 1} - 1 $$ i $$g(x) = \frac{x^2}{\sqrt{x^2 + 1} + 1}$$ obliczające to samo z punktu widzenia matematyki. \\Ponieważ w funkcji $f(x)$, x dąży do $0$ to wartość pierwiastka dąży do 1. W rezultacie odejmujemy od siebie bardzo bliskie liczby i przez stosowane obcięcie wynik szybko traci na dokładności. \\W funkcji $g(x)$, błąd odejmowania już nie występuje dzięki czemu mamy zachowaną wyższą precyzję obliczeń, mimo że wyrażenie jest równoważne.
  \begin{table}[ht]
    \centering
    \begin{tabular}{|c|c|c|c|}
      \hline
      n & $x = 8^{-n}$ & $f(x)$ & $g(x)$ \\
      \hline
      \hline
      0 & 1.0 & 0.41421356237309515 & 0.4142135623730951 \\
      1 & 0.125 & 0.0077822185373186414 & 0.0077822185373187065 \\
      2 & 0.015625 & 0.00012206286282867573 & 0.00012206286282875901 \\
      3 & 0.001953125 & 1.9073468138230965e-6 & 1.907346813826566e-6 \\
      4 & 0.000244140625 & 2.9802321943606103e-8 & 2.9802321943606116e-8 \\
      5 & 3.0517578125e-5 & 4.656612873077393e-10 & 4.6566128719931904e-10 \\
      6 & 3.814697265625e-6 & 7.275957614183426e-12 & 7.275957614156956e-12 \\
      7 & 4.76837158203125e-7 & 1.1368683772161603e-13 & 1.1368683772160957e-13 \\
      8 & 5.960464477539063e-8 & 1.7763568394002505e-15 & 1.7763568394002489e-15 \\
      9 & 7.450580596923828e-9 & 0.0 & 2.7755575615628914e-17 \\
      10 & 9.313225746154785e-10 & 0.0 & 4.336808689942018e-19 \\
      11 & 1.1641532182693481e-10 & 0.0 & 6.776263578034403e-21 \\
      12 & 1.4551915228366852e-11 & 0.0 & 1.0587911840678754e-22 \\
      13 & 1.8189894035458565e-12 & 0.0 & 1.6543612251060553e-24 \\
      14 & 2.2737367544323206e-13 & 0.0 & 2.5849394142282115e-26 \\
      15 & 2.842170943040401e-14 & 0.0 & 4.0389678347315804e-28 \\
      16 & 3.552713678800501e-15 & 0.0 & 6.310887241768095e-30 \\
      17 & 4.440892098500626e-16 & 0.0 & 9.860761315262648e-32 \\
      18 & 5.551115123125783e-17 & 0.0 & 1.5407439555097887e-33 \\
      19 & 6.938893903907228e-18 & 0.0 & 2.407412430484045e-35 \\
      20 & 8.673617379884035e-19 & 0.0 & 3.76158192263132e-37 \\
      \hline
    \end{tabular}
    \caption{wartości z wyjścia programu \texttt{zadanie5.jl} (ograniczone do 21 pierwszych wyników)}
  \end{table}
  \section*{zadanie 7}

  \begin{table}[ht]
    \centering
    \begin{tabular}{|c|c|c|c|}
      \hline
      & & & \\
      \emph{n} & $ \widetilde{f'}(1.0) , h = 2^{-n}$ & $f'(1.0)$ & $|\widetilde{f'}(1.0) - f'(1.0)|$ \\
      & & & \\
      \hline
      \hline
      0 & 0.8218544151266947 & 0.11694228168853815 & 0.7049121334381565 \\
      1 & 0.7771241507177988 & 0.11694228168853815 & 0.6601818690292607 \\
      2 & 0.746269987940468 & 0.11694228168853815 & 0.6293277062519298 \\
      3 & 0.7278860798075346 & 0.11694228168853815 & 0.6109437981189965 \\
      4 & 0.7178180618879821 & 0.11694228168853815 & 0.600875780199444 \\
      5 & 0.7125457423057711 & 0.11694228168853815 & 0.5956034606172329 \\
      6 & 0.7098474639715988 & 0.11694228168853815 & 0.5929051822830607 \\
      7 & 0.7084824701063042 & 0.11694228168853815 & 0.591540188417766 \\
      8 & 0.7077959684277175 & 0.11694228168853815 & 0.5908536867391794 \\
      9 & 0.7074517112415606 & 0.11694228168853815 & 0.5905094295530224 \\
      10 & 0.7072793304150764 & 0.11694228168853815 & 0.5903370487265382 \\
      11 & 0.7071930768624952 & 0.11694228168853815 & 0.5902507951739571 \\
      12 & 0.7071499342910101 & 0.11694228168853815 & 0.590207652602472 \\
      13 & 0.7071283590557869 & 0.11694228168853815 & 0.5901860773672487 \\
      14 & 0.7071175704477355 & 0.11694228168853815 & 0.5901752887591973 \\
      15 & 0.7071121758999652 & 0.11694228168853815 & 0.5901698942114271 \\
      16 & 0.7071094785642345 & 0.11694228168853815 & 0.5901671968756963 \\
      17 & 0.7071081298636273 & 0.11694228168853815 & 0.5901658481750891 \\
      18 & 0.7071074555278756 & 0.11694228168853815 & 0.5901651738393374 \\
      19 & 0.7071071183308959 & 0.11694228168853815 & 0.5901648366423577 \\
      20 & 0.7071069497615099 & 0.11694228168853815 & 0.5901646680729717 \\
      21 & 0.7071068654768169 & 0.11694228168853815 & 0.5901645837882787 \\
      22 & 0.7071068231016397 & 0.11694228168853815 & 0.5901645414131016 \\
      23 & 0.707106800749898 & 0.11694228168853815 & 0.5901645190613598 \\
      24 & 0.7071067914366722 & 0.11694228168853815 & 0.5901645097481341 \\
      25 & 0.7071067839860916 & 0.11694228168853815 & 0.5901645022975535 \\
      26 & 0.7071067690849304 & 0.11694228168853815 & 0.5901644873963923 \\
      27 & 0.7071067690849304 & 0.11694228168853815 & 0.5901644873963923 \\
      28 & 0.7071067690849304 & 0.11694228168853815 & 0.5901644873963923 \\
      29 & 0.7071067094802856 & 0.11694228168853815 & 0.5901644277917475 \\
      30 & 0.7071065902709961 & 0.11694228168853815 & 0.5901643085824579 \\
      31 & 0.7071065902709961 & 0.11694228168853815 & 0.5901643085824579 \\
      32 & 0.7071065902709961 & 0.11694228168853815 & 0.5901643085824579 \\
      33 & 0.7071056365966797 & 0.11694228168853815 & 0.5901633549081415 \\
      34 & 0.7071037292480469 & 0.11694228168853815 & 0.5901614475595087 \\
      35 & 0.7070999145507812 & 0.11694228168853815 & 0.5901576328622431 \\
      36 & 0.7071075439453125 & 0.11694228168853815 & 0.5901652622567743 \\
      37 & 0.70709228515625 & 0.11694228168853815 & 0.5901500034677118 \\
      38 & 0.70709228515625 & 0.11694228168853815 & 0.5901500034677118 \\
      39 & 0.70703125 & 0.11694228168853815 & 0.5900889683114618 \\
    \end{tabular}
  \end{table}

  \begin{table}[ht]
    \centering
    \begin{tabular}{|c|c|c|c|}
      
      40 & 0.70703125 & 0.11694228168853815 & 0.5900889683114618 \\
      41 & 0.70703125 & 0.11694228168853815 & 0.5900889683114618 \\
      42 & 0.70703125 & 0.11694228168853815 & 0.5900889683114618 \\
      43 & 0.70703125 & 0.11694228168853815 & 0.5900889683114618 \\
      44 & 0.70703125 & 0.11694228168853815 & 0.5900889683114618 \\
      45 & 0.703125 & 0.11694228168853815 & 0.5861827183114618 \\
      46 & 0.703125 & 0.11694228168853815 & 0.5861827183114618 \\
      47 & 0.6875 & 0.11694228168853815 & 0.5705577183114618 \\
      48 & 0.6875 & 0.11694228168853815 & 0.5705577183114618 \\
      49 & 0.625 & 0.11694228168853815 & 0.5080577183114618 \\
      50 & 0.5 & 0.11694228168853815 & 0.38305771831146185 \\
      51 & 0.5 & 0.11694228168853815 & 0.38305771831146185 \\
      52 & 0.0 & 0.11694228168853815 & 0.11694228168853815 \\
      53 & 0.0 & 0.11694228168853815 & 0.11694228168853815 \\
      54 & 0.0 & 0.11694228168853815 & 0.11694228168853815 \\
      \hline
    \end{tabular}
    \caption{wartości z wyjścia programu \texttt{zadanie5.jl}}
  \end{table}

  Program oblicza pochodną funkcji w przybliżeniu $ \widetilde{f'}(x) = \frac{f(x + h) - f(x)}{h}$ w  punkcie $x=1$ oraz dokładną wartość pochodnej $f'(x) = cos(x) - 3sin(3x)$\\
  Dla dużych n, $h \rightarrow 0$, czyli w liczniku dostajemy bardzo zbliżone wartości i przez obcięcie odejmowania dostajemy 0, z tego samego powodu od pewnego momentu błąd ma prawie stałą wartość, ok 0.59.
\end{document}