\section{zadanie 1}
Zadanie polegało na niewielkiej zmianie wartości \(x_4\) oraz \(x_5\).

\subsection{Wnioski:}
Zmiany wyników są widoczne przy obliczeniach w arytmetyce \textbf{Float64} (Double). Niewielka zmiana danych wejściowych znacząco wpłynęła na wynik działania algorytmów. W tym przypadku, mimo że ilorazy miały różny rząd wielkości, kolejność sumowania nie miała, w arytmetyce \textbf{Float64}, znaczenia dla wyników. Dla arytmetyki \textbf{Float32} zmiana nie miała wpływu na wyniki.

\begin{table}[ht]
  \centering
  \begin{tabular}{|c|c|}
    \hline
    \emph{typ zmiennej} & \emph{output} \\
    \hline
    \hline
    Float32 & -0.4999443 \\
     & -0.4543457 \\
     & -0.5 \\
     & -0.5 \\
    Float64 & -0.004296342739891585 \\
     & -0.004296342998713953 \\
     & -0.004296342842280865 \\
     & -0.004296342842280865 \\
    \hline
  \end{tabular}
  \caption{wartości z wyjścia programu \texttt{zadanie1.jl}}
\end{table}