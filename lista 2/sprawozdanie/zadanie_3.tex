\section{zadanie 3}
Zadanie polegało na rozwiązaniu układu równań linowych \textbf{\textit{Ax = b}} za pomocą dwóch algorytmów: eliminacji Gaussa \( x = A \backslash b \) oraz \( x = A^{-1}b \), gdzie \( x = (1,...,1)^T \), 
\( A \in \mathbb{R}^{n \times n} \) jest losową macierzą lub macierzą Hilberta,
\( b \in \mathbb{R}^n\) jest wektorem prawych stron\\
Przedstawione wyniki \emph{gauss err} i \emph{inverted err} zostały policzone korzystając ze wzoru: \[\frac{norm(\tilde{x} - x)}{norm(x)} \]. Eksperyment pokazuje że dla macierzy Hilberta zadanie jest źle uwarunkowane przez błędy obliczeń, niezależnie od wybranej metody. 
\begin{table}[ht]
  \centering
  \makebox[\textwidth]{
    \begin{tabular}{|c|c|c|c|c|}
      \hline
      \multicolumn{5}{|c|}{Dla macierzy Hilberta} \\
      \hline
      \emph{n} & rank(A) & cond(A) & \emph{gauss err} & \emph{inverted err}  \\
      \hline
      \hline
      1 & 1 & 1.0 & 0.0 & 0.0 \\
      3 & 3 & 524.0567775860644 & 8.022593772267726e-15 & 0.0 \\
      5 & 5 & 476607.25024259434 & 1.6828426299227195e-12 & 3.3544360584359632e-12 \\
      7 & 7 & 4.75367356583129e8 & 1.2606867224171548e-8 & 4.713280397232037e-9 \\
      9 & 9 & 4.931537564468762e11 & 3.8751634185032475e-6 & 4.541268303176643e-6 \\
      11 & 10 & 5.222677939280335e14 & 0.00015827808158590435 & 0.007618304284315809 \\
      13 & 11 & 3.344143497338461e18 & 0.11039701117868264 & 5.331275639426837 \\
      15 & 12 & 3.674392953467974e17 & 4.696668350857427 & 7.344641453111494 \\
      17 & 12 & 1.263684342666052e18 & 13.707236683836307 & 10.516942378369349 \\
      19 & 13 & 6.471953976541591e18 & 9.720589712655698 & 12.233761393757726 \\
      21 & 13 & 3.290126328601399e18 & 56.40267595616145 & 43.4753048667801 \\
      23 & 13 & 6.313778670724671e17 & 12.483655076018373 & 13.803784630487236 \\
      25 & 13 & 1.3719347461445998e18 & 10.15919484338797 & 16.93987792970947 \\
      27 & 14 & 4.424587877361583e18 & 30.11850661319111 & 28.752075126924804 \\
      29 & 14 & 8.05926200352767e18 & 25.047149256115667 & 95.60461031775714 \\
      \hline
    \end{tabular}
  }
  \caption{wartości z wyjścia programu \texttt{zadanie3.jl}}
\end{table}

\begin{table}[th]
  \makebox[\textwidth]{
    \centering
    \begin{tabular}{|c|c|c|c|c|c|}
    \hline
    \multicolumn{6}{|c|}{Dla macierzy losowej} \\
    \hline
    \emph{n} & \emph{c} & \emph{rank(A)} & \emph{cond(A)} & \emph{gauss err} & \emph{inverted err}  \\
    \hline
    \hline
    5 & 1.0 & 5 & 1.0000000000000004 & 1.4895204919483638e-16 & 1.1102230246251565e-16 \\
    5 & 10.0 & 5 & 10.000000000000005 & 9.930136612989092e-17 & 1.9860273225978183e-16 \\
    5 & 1000.0 & 5 & 999.9999999999449 & 2.106710720622688e-14 & 2.1518860363986142e-14 \\
    5 & 1.0e7 & 5 & 9.999999986840717e6 & 3.3468138832372843e-10 & 3.2251456699322597e-10 \\
    5 & 1.0e12 & 5 & 1.0000120160444469e12 & 2.9337670118852183e-5 & 3.466970143890919e-5 \\
    5 & 1.0e16 & 4 & 9.058833363884072e15 & 0.2872201265139816 & 0.2867517436738616 \\
    10 & 1.0 & 10 & 1.0000000000000016 & 2.9996574304705467e-16 & 2.0770370905276122e-16 \\
    10 & 10.0 & 10 & 10.000000000000004 & 3.2177320244274193e-16 & 4.1093252186201184e-16 \\
    10 & 1000.0 & 10 & 1000.0000000000507 & 8.944843056870317e-15 & 8.39316858142599e-15 \\
    10 & 1.0e7 & 10 & 9.999999998732708e6 & 4.955625093760456e-11 & 5.707106429696928e-11 \\
    10 & 1.0e12 & 10 & 9.999449703833724e11 & 3.3926252911339658e-6 & 1.983321604069273e-6 \\
    10 & 1.0e16 & 9 & 9.231859885643694e15 & 0.012840193015659251 & 0.026087199776662497 \\
    20 & 1.0 & 20 & 1.0000000000000018 & 4.946414258176871e-16 & 4.3568297570458958e-16 \\
    20 & 10.0 & 20 & 9.999999999999993 & 4.1910000110727263e-16 & 5.1238927586247155e-16 \\
    20 & 1000.0 & 20 & 1000.0000000000015 & 1.5451217583057598e-14 & 1.26308058855971e-14 \\
    20 & 1.0e7 & 20 & 9.999999995287322e6 & 1.2373756600196829e-11 & 3.903576264676482e-11 \\
    20 & 1.0e12 & 20 & 1.0000365381020696e12 & 1.7519210494067048e-5 & 1.5030584949375567e-5 \\
    20 & 1.0e16 & 19 & 1.07738779481172e16 & 0.034284522301952836 & 0.052557188402976115 \\
    \hline
    \end{tabular}
    }
  \caption{wartości z wyjścia programu \texttt{zadanie3.jl}}
\end{table}
