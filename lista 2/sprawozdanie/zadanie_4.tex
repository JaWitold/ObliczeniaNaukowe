\section{zadanie 4}
Zadanie polegało na instalacji pakietu \textbf{Polynomials} oraz wykorzystaniu jego funkcji do znalezienia miejsc zerowych wielomianu Wilkinsona w dwóch postaciach - naturalnej oraz iloczynowej:\\

\( P(x) = x^{20} - 210x^{19} + 20615x^{18} - 1256850x^{17} + 53327946x^{16} - 1672280820x^{15} + 40171771630x^{14} - 756111184500x^{13} + 11310276995381x^{12} - 135585182899530x^{11} + 1307535010540395x^{10} - 10142299865511450x^9 + 63030812099294896x^8 - 311333643161390640x^7 + 1206647803780373360x^6 - 3599979517947607200x^5 + 8037811822645051776x^4 - 12870931245150988800x^3 + 13803759753640704000x^2 - 8752948036761600000x^1 + 2432902008176640000x^0 \) \\
oraz
\[ p(x) = \prod_{i=1}^{20}(x-i)\]\\

\subsection{Wnioski:}
Oczywiście \(p(x) = P(x)\). Obliczone pierwiastki P(x) różnią się od spodziewanych wyników. Niewielka zmiana współłczynnika znacząco zmienia wyniki co implikuje, że zadanie jest źle uwarunkowane. Dokładność arytmetyki \textbf{Float64} ma 15-17 cyfr znaczących w systemie dziesiętnym. Wolny współczynnik wielomianu to \(20!\) i ma 19 cyfr znaczących co wykracza poza precyzję jaką dysponujemy.

\begin{table}
  \makebox[\textwidth]{
    \centering
    \begin{tabular}{|c|c|c|c|c|}
    \hline
    \emph{k} & \(z_k\) & \(|P(z_k)|\) & \(|p(z_k)|\) & \(|z_k - k|\)  \\
    \hline
    \hline
    1 & 0.9999999999996989 & 35696.50964788257 & 36720.50964788227 & 3.0109248427834245e-13 \\
    2 & 2.0000000000283182 & 176252.60026668405 & 192636.60026691604 & 2.8318236644508943e-11 \\
    3 & 2.9999999995920965 & 279157.6968824087 & 362101.69687113096 & 4.0790348876384996e-10 \\
    4 & 3.9999999837375317 & 3.0271092988991085e6 & 2.7649652999648857e6 & 1.626246826091915e-8 \\
    5 & 5.000000665769791 & 2.2917473756567076e7 & 2.2277473671348542e7 & 6.657697912970661e-7 \\
    6 & 5.999989245824773 & 1.2902417284205095e8 & 1.2769707122070245e8 & 1.0754175226779239e-5 \\
    7 & 7.000102002793008 & 4.805112754602064e8 & 4.780526156335614e8 & 0.00010200279300764947 \\
    8 & 7.999355829607762 & 1.6379520218961136e9 & 1.6337585675856934e9 & 0.0006441703922384079 \\
    9 & 9.002915294362053 & 4.877071372550003e9 & 4.870348427548107e9 & 0.002915294362052734 \\
    10 & 9.990413042481725 & 1.3638638195458128e10 & 1.362843071072106e10 & 0.009586957518274986 \\
    11 & 11.025022932909318 & 3.585631295130865e10 & 3.584087897760478e10 & 0.025022932909317674 \\
    12 & 11.953283253846857 & 7.533332360358197e10 & 7.531256581876213e10 & 0.04671674615314281 \\
    13 & 13.07431403244734 & 1.9605988124330817e11 & 1.9602984002587503e11 & 0.07431403244734014 \\
    14 & 13.914755591802127 & 3.5751347823104315e11 & 3.574748406282602e11 & 0.08524440819787316 \\
    15 & 15.075493799699476 & 8.21627123645597e11 & 8.215740477766903e11 & 0.07549379969947623 \\
    16 & 15.946286716607972 & 1.5514978880494067e12 & 1.5514314565843672e12 & 0.05371328339202819 \\
    17 & 17.025427146237412 & 3.694735918486229e12 & 3.6946500070912217e12 & 0.025427146237412046 \\
    18 & 17.99092135271648 & 7.650109016515867e12 & 7.650001670877033e12 & 0.009078647283519814 \\
    19 & 19.00190981829944 & 1.1435273749721195e13 & 1.14351402511197e13 & 0.0019098182994383706 \\
    20 & 19.999809291236637 & 2.7924106393680727e13 & 2.7923942556843e13 & 0.00019070876336257925 \\
    \hline
    \end{tabular}
    }
    \caption{wartości z wyjścia programu \textbf{zadanie4.jl, b)}}

\end{table}

\begin{table}[th]
  \makebox[\textwidth]{
    \centering
    \begin{tabular}{|c|c|c|c|}
    \hline
    \emph{k} & \(z_k\) & \(|P'(z_k)|\) & \(|z_k - k|\) \\
    \hline
    \hline
    1 & 0.9999999999998357 + 0.0im & 20259.872313418207 & 1.6431300764452317e-13 \\
    2 & 2.0000000000550373 + 0.0im & 346541.4137593836 & 5.503730804434781e-11 \\
    3 & 2.99999999660342 + 0.0im & 2.2580597001197007e6 & 3.3965799062229962e-9 \\
    4 & 4.000000089724362 + 0.0im & 1.0542631790395478e7 & 8.972436216225788e-8 \\
    5 & 4.99999857388791 + 0.0im & 3.757830916585153e7 & 1.4261120897529622e-6 \\
    6 & 6.000020476673031 + 0.0im & 1.3140943325569446e8 & 2.0476673030955794e-5 \\
    7 & 6.99960207042242 + 0.0im & 3.939355874647618e8 & 0.00039792957757978087 \\
    8 & 8.007772029099446 + 0.0im & 1.184986961371896e9 & 0.007772029099445632 \\
    9 & 8.915816367932559 + 0.0im & 2.2255221233077707e9 & 0.0841836320674414 \\
    10 & 10.095455630535774 - 0.6449328236240688im & 1.0677921232930157e10 & 0.6519586830380407 \\
    11 & 10.095455630535774 + 0.6449328236240688im & 1.0677921232930157e10 & 1.1109180272716561 \\
    12 & 11.793890586174369 - 1.6524771364075785im & 3.1401962344429485e10 & 1.665281290598479 \\
    13 & 11.793890586174369 + 1.6524771364075785im & 3.1401962344429485e10 & 2.0458202766784277 \\
    14 & 13.992406684487216 - 2.5188244257108443im & 2.157665405951858e11 & 2.518835871190904 \\
    15 & 13.992406684487216 + 2.5188244257108443im & 2.157665405951858e11 & 2.7128805312847097 \\
    16 & 16.73074487979267 - 2.812624896721978im & 4.850110893921027e11 & 2.9060018735375106 \\
    17 & 16.73074487979267 + 2.812624896721978im & 4.850110893921027e11 & 2.825483521349608 \\
    18 & 19.5024423688181 - 1.940331978642903im & 4.557199223869993e12 & 2.4540214463129764 \\
    19 & 19.5024423688181 + 1.940331978642903im & 4.557199223869993e12 & 2.0043294443099486 \\
    20 & 20.84691021519479 + 0.0im & 8.756386551865696e12 & 0.8469102151947894 \\
    \hline
    \end{tabular}
    }
  \caption{wartości z wyjścia programu \textbf{zadanie4.jl, b)}}
\end{table}
