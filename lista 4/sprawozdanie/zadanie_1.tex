\section{zadanie 1}
Zadanie polegało na implementacji funkcji obliczającej ilorazy różnicowe.

\subsection{Iloraz różnicowy: }
Jeśli \(f: X \rightarrow Y\) oraz \(x_0, x_1 \in X\) to ilorazem różnicowym nazywamy
\[\ f[x_0] = f(x_0)\]
\[\ f[x_0, x_1] = \frac{f(x_0) - f(x_1)}{x_0 - x_1}\]
Ponadto ilorazy różnicowe spełniają równość:
\[ f[x_0, x_1, \ldots, x_n] = \frac{f[x_1, x_2, \ldots, x_{n-1}] - f[x_0, x_1, \ldots, x_n]}{x_n - x_0} \]

Warto zauważyć że jeśli \(\Delta x = x_0 - x_1 \rightarrow 0 \) to iloraz różnicowy \(f[x_0, x_1]\) odpowiada wartości pochodnej pierwszego stopnia w punkcie \(x_0\).

\subsection{Opis algorytmu: }
Algorytm na wejściu dostaje wektory długości \(n + 1\) argumentów i wartości pewnej funkcji.\\ Wyjściem jest wektor ilorazów różnicowych, długości \(n + 1\), w postaci: 
\[ [f[x_0], f[x_0, x_1], \ldots,f[x_0, x_1, \ldots, x_{n - 1}], f[x_0, x_1, \ldots, x_n] ]\] 
Funkcja została oprogramowana bez użycia macierzy 2-wymiarowej. 
Aby to osiągnąć korzystamy z zagnieżdżonej pętli oraz z równości obliczania ilorazów różnicowych.

\begin{align*}
  f[x_0] &\searrow & & & & & f[x_0]\\
  f[x_1] & \begin{array}{c} \rightarrow \\ \searrow \end{array} &f[x_0, x_1]&\searrow & & & f[x_0, x_1]\\
  \vdots & & & &\cdots & &\vdots\\
  f[x_{n - 1}] &\begin{array}{c} \rightarrow \\ \searrow \end{array} &f[x_{n-2}, x_{n-1}] &\begin{array}{c} \rightarrow \\ \searrow \end{array} &\cdots &\begin{array}{c} \rightarrow \\ \searrow \end{array} &f[x_{0}, x_{1}, \ldots, x_{n-1}]\\
  f[x_n] &\rightarrow & f[x_{n-1}, x_{n}] & \rightarrow &\cdots &\rightarrow &f[x_{0}, x_{1}, \ldots, x_n]
\end{align*}

